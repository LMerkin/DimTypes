% vim:ts=2:syntax=tex
\documentclass[11pt,a4paper]{article}
\usepackage{times}
\usepackage{fullpage}

\title{\textbf{Implementation of Dimension Types in C++0x}}

\author{\textbf{Dr Leonid A. Timochouk} \\[3mm]
				FICC Core Libraries, UBS Investment Bank   \\
				Europastrasse 1, 8152 Opfikon, Switzerland \\[3mm]
				\texttt{L.Timochouk@gmail.com}}

\date{2 July 2011}
\renewcommand{\thesubsection}{\alph{subsection}}

\begin{document}
\maketitle

\section{Introduction}
\label{Intro}

\subsection{FIG}

The idea of dimension types is nothing new (see Section~\ref{Overview} for a
historical overview). Informally speaking, \textit{dimension types} is a
feature of a type system of a compiled programming language which allows the
user to express the physical dimensions and units of quantities
represented by entities of that language, and to enforce, by means of static
type-checking performed at compile-time, the correct structure of dimensioned
expressions.

For example, if the constant \texttt{c} represents the speed of light in the
vacuum, then the C/C++ definition like
\begin{verbatim}
double const c = 299792458.0;
\end{verbatim}
only tells the compiler (and the programmer) that the machine representation
of $c$ is \texttt{double} and that $c$ has the constant value given above.
However, what we would really like to express is that $c$ is a physical
quantity, its dimension is that of the velocity (i.~e., $ l t^{-1} $,
where $l$ is length and $t$ is time), and its physical unit is
$ km\ sec^{-1} $.

Capturing dimension and unit properties of physical quantities in a programming
language is indeed important because this would preclude erroneous operations
with such dimensioned quantities. For example, if we define another quantity to
represent the astronomical unit (roughly speaking, this is the mean distance in
$km$ between Earth and Sun),
\begin{verbatim}
double const AU = 149597870.6996262;
\end{verbatim}
then expressions like \texttt{AU / c} and even \texttt{AU * c} are formally
valid. The former represents the time $t$ required for light to travel the
distance of $ 1 AU $, and the latter has no straightforward physical
interpretation (its dimension is $ l^2\ t^{-1} $ and the unit is
$ km^2\ sec^{-1} $) but it is nevertheless well-formed.

On the other hand, the expression \texttt{AU + c} is a valid C/C++ expression
but it is clearly ill-formed from the physical point of view (contains addition
of quantities of different physical dimensions). Furthermore, if \texttt{AU} is
expressed in meters rather than in kilometers:
\begin{verbatim}
double const AUm = 149597870699.6262;
\end{verbatim}
then expressions \texttt{AUm / c} and \texttt{AUm * c} also become ill-formed
because their operands are expressed in incompatible units.

Generally speaking, any \textit{monomial} expressions over dimensioned
quantities (multiplication, division, raising to a rational power) are
well-formed provided that the operands' units are the same. However,
\textit{polynomial} expressions (sums and differences) are well-formed if and
only if all their monomial terms have same physical dimensions and units.
Furthermore, any non-polynomial functions (e.~g.\ \texttt{exp}, \texttt{sin},
\ldots) can only be applied to \textit{dimension-less} quantities, because
otherwise the Taylor power series of such functions, viewed as polynomials of
infinite degree, would be ill-formed. For example, the expression $\exp c$
where $c$ is the speed of light defined above is ill-formed, whereas
$\exp\left(\frac{c}{v}\right)$ where $v$ is another velocity expressed in
$ km\ sec^{-1} $ is well-formed because $\frac{c}{v}$ is dimension-less.

However, the standard type systems of all mainstream programming languages
(C, C++, Java, FORTRAN, Ada, OCaml, \ldots) do not allow us to express the
dimensions and units of physical quantities directly. As a result, it
would not be possible to capture errors originating from invalid dimensions
and/or units at compile-time (or indeed at run-time, before such an error
manifests itself as a software failure).

This is indeed an important software engineering problem which particularly
affects scientific and engineering computing and real-time control systems. For
example, the loss of Mars Climate Orbiter spacecraft in 1999 was caused by a
software error which was due to mix-up of metric and imperial units across
the ground and on-board software components~\cite{MCOReport1999}. The failure
was attributed by the NASA investigation board to various management
shortcomings in the software development and testing cycle. However, the
underlying cause of this high-profile mishap was the inability of the
programming language used (apparently FORTRAN) to capture dimension / unit
errors through its type system.

\section{Requirements for a dimensioned type system}

It is desirable that a system of dimension types implemented for a programming
language $X$ would possess the following properties.

\textit{First}, $X$ itself should be an accepted mainstream programming
language used sufficiently widely for scientific, engineering and real-time
computations, as these application areas would benefit most from the advent of
dimension types.

\textit{Second}, dimension types should preferably be implemented in terms of
the standard type system of language $X$ without introducing any new syntactic
constructs or annotations. Consequently, type-checking of dimensioned
expressions should be performed by the $X$ compiler as part of normal
type-checking process at compile-time, without the recourse to any external
tools and without the need to modify the compiler itself.

\textit{Third}, dimension types should provide zero run-time overhead, either
temporal or spatial, compared to representing the corresponding quantities
solely in terms of underlying elementary data types (such \texttt{double} in
C/C++). In other words, the whole complexity of dimensioned type system must
be absorbed at compilation time only.

\textit{Fourth}, the system of dimension types must allow for monomial
expressions over dimensioned quantities containing arbitrary \textit{rational}
(not just integral) powers. For example, a monomial expression like
$ c^\frac{1}{2} AU^\frac{3}{4} $ (see Section~\ref{Intro}) should be accepted.

\textit{Fifth}, it should be possible to express dimensioned quantities in
multiple units, to convert them automatically between different units and to
check the consistency of units used as part of the over-all type-checking
process.

For example, for the dimension $l$ (length) the units could be meters,
kilometers and so on. Continuing with the above example, the expression
$ c^\frac{1}{2} AU^\frac{3}{4} $ has dimension
$ l^\frac{5}{4} t^\frac{-1}{2} $ where $t$ is time. This expression is
well-formed if only if the unit of length used in both $c$ and $AU$ is the same.
If that unit was, for example, kilometer but we now want to change it to
meter, than the value of any expression of dimension proportional to
$ l^\frac{5}{4} $ would be multiplied by the factor $ 1000^\frac{5}{4} $ and
this should be performed automatically via a dimensioned type conversion
operator. A similar operation would be applied when the unit of time $t$
changes.

An alternative approach is to maintain just one fundamental unit for each
dimension and to automatically convert all dimensioned quantities to the
fundamental units at construction time. However, in our view, this approach has
two disadvantages:
\begin{itemize}
\item it would enforce data representation which may be physically
	counter-intuitive (for example, the fundamental unit for length $l$ is meter
	but in nuclear physics, microscopic units of length are used instead);
\item it may also incur run-time overhead from unnecessary implicit data
	conversions.
\end{itemize}

\textit{In addition}, there is a problem of configurability of the set of
dimensions and units. In physics, there are 7 fundamental dimensions which are
standardised in the SI system and considered to be irreducible with respect to
each other~\cite{SI}:

\begin{table}[ht!]
\begin{center}
\begin{tabular}{||l|l||}
\hline
Dimension												& Fundamental unit (SI)	\\ \hline \hline
Length ($l$)										& meter  ($m$)					\\ \hline
Time ($t$)											& second ($sec$)				\\ \hline
Mass ($m$)											& kilogram ($kg$)				\\ \hline
Electrical current ($I$)				& Ampere ($A$)					\\ \hline
Thermodynamic temperature ($T$)	&	Kelvin ($K$)					\\ \hline
Substance amount ($n$)					& mole ($mol$)					\\ \hline
Luminous intensity ($I_v$)			& candela ($cd$)				\\ \hline
\end{tabular}
\caption{Fundamental physical dimensions and units.\label{TabFund}}
\end{center}
\end{table}

It is therefore recommended that any implementation of dimension types should
provide for \textit{at least} the fundamental dimensions and units specified in
Table~\ref{TabFund}. However, there are several issues here which should be
noted.

First of all, the 7 fundamental dimensions are not entirely independent from
each other; they are connected, for example, via the fundamental physical
constants (such as $c$). Some implementations of dimension types try to take
such connections into account as discussed in Section~\ref{Overview} below,
though in our view, this is not strictly necessary and is more confusing than
useful.

Similarly, one can argue that the substance amount $n$ is not really a
fundamental dimension but a dimension-less quantity because it is in fact
proportional to the number of molecules of a given substance; nevertheless, it
has been internationally standardised in the SI system as a fundamental
dimension and the standard should be adhered to.

More importantly, the user of a dimensioned type system may want to introduce
their own dimensions. For example, although the \textit{angle} and
\textit{solid angle} (their fundamental units being radian and steradian,
respectively) are considered in the SI system to be dimension-less quantities,
it is often useful to make them into stand-alone dimensions, for example in
order to be able to express them in other conventional units as well
(such as degrees and square degrees), whereas dimension-less quantities have no
units. In financial computations, it would be useful to have \textit{money} as
yet another dimension, e.~g.\ with US\$ being its fundamental unit; and so on.

It would therefore be advantageous if a system of dimension types allows the
user to declare new dimensions apart from those listed in Table~\ref{TabFund},
and also to create arbitrary units of any dimensions. Unfortunately, it appears
that this requirement would typically lead to significant complications in the
implementation of dimension types and would conflict with other requirements
presented above. For this reason, this feature is currently not supported by
the existing implementations we are aware of. Typically, a dimensioned type
system would provide a pre-configured set of dimensions (though it may be
larger than the 7 fundamental SI dimensions) and a pre-configured set of units
for each dimension (or just one fundamental unit).

\section{Overview of existing research}
\label{Overview}

Dimensional analysis has been used extensively in theoretical physics (see for
example~\cite{SedovDim,Ovsyan}) in order to provide approximate values for
various physical quantities (especially in nuclear physics) or to reduce
the dimensionality of partial differential equations of mathematical physics,
possibly simplifying them down to ordinary differential equations (construction
of self-similar solutions, e.~g.\ in continuous media mechanics or in
mathematical finance).

In computer science, the idea of augmenting the type systems of programming
languages by information on physical dimensions of represented quantities was
first developed in late 1970s -- early 1980s. A systematic theory of dimension
types (including formal type inference rules) was provided
in~\cite{Kennedy1994,Kennedy1997} and a software implementation of dimension
types was constructed within the ML Kit compiler of Standard ML. However,
there are three problems with this approach:
\begin{itemize}
\item only the dimensions of quantities were considered, not the units; more
			precisely, only one unit for each dimension was allowed, so the spaces
			of dimensions and units were isomorphic;
\item implementation of dimension types in ML Kit required modifications to the
			compiler which is undesirable from the practical point of view;
\item ML Kit and Standard ML are not widely used in scientific and engineering
			computations which hindered integration of dimension types into software
			engineering practices.
\end{itemize}

More recently, dimension types have been implemented in
Haskell~\cite{HaskellDims}. This implementation is based only on the native
Haskell type system features (such as multi-parameter type classes, phantom
types, functional dependencies etc) supported by Glasgow Haskell Compiler.
However, it still has some limitations:
\begin{itemize}
\item arbitrary rational powers of dimensioned quantities are not supported;
			the only allowed denominators are 1 and 2;
\item multiple units for each dimension are allowed, but all such units are
			obtained from one fundamental unit by applying various SI decimal
			prefixes.
\end{itemize}

Recent versions of computer algebra systems Maple~\cite{MapleDims} and
Mathematica~\cite{MathematicaDims} are equipped with rather comprehensive
modules for dimensional and unit analysis. However, Maple and Mathematica
programming languages are interpreted and dynamically-typed, so their
dimensional analysis tools are essentially outside the definition of
dimension types as presented in Section~\ref{Intro}. Furthermore, the primary
purpose of these systems is symbolic and algebraic computations; they are less
well optimised for heavy-duty numerical computations (scientific, engineering,
real-time) than the mainstream compiled programming languages.

Of the general-purpose programming languages, the (arguably) most suitable one
for supporting dimension types is C++. This is due to C++ template
meta-programming mechanisms which provide a Turing-complete
\textit{compile-time} evaluation engine~\cite{C++Turing}. In particular,
parameterisation of (templated) C++ types can be used to represent dimension and
unit information. There exist several dimension types implementations in
C++:~\cite{SIunits}.




\bibliographystyle{acm}
\bibliography{DimTypesC++0x}

\end{document}
